% --*- coding:utf-8-unix mode:latex -*--
%\include{Begin}
%%%%%%%%%%%%%%%%%%%%%%%%%%%%%%%%%%%%%%%%%%%%%%%%%%%%%%%%%%%%%%%%%%%%%%%%%%%%%%%

\section{荷物の搬入}

\subsection{日時・場所}

\begin{tabular}{p{2zw}rp{38zw}}
  日時 & : & 2020年4月6日(月) 7:50$\sim$ 8:30\\
  場所 & : & ????, 東ロータリー            %教室決まったら変更、多分K101
\end{tabular}

\subsection{目的}
迅速に荷物を運ぶ!!!!

\subsection{タイムスケジュール}
% 時刻は必ず4桁(00:00)で書くこと!!!
\begin{longtable}{p{3zw}p{39zw}}
   7:10 & \textbf{◎ 搬入開始} \\
        & \ \  \textbullet \ \ ??,??,??,??は車を東ロータリーに停めておく \\        %車係を書く、要変更
        & \ \  \textbullet \ \ ????から物品を車に積み込む \\\\                               %教室決まったら変更

   7:20 & \textbf{◎ 物品確認} \\
        & \ \  \textbullet \ \ 先遣隊,後遣隊以外のスタッフはK101へ戻って読み合わせを始める \\
        & \ \  \textbullet \ \ 先遣隊,後遣隊は,本企画書と搬入した物品を確認する \\
        & \ \  \textbullet \ \ 物品がない場合は先遣隊が購入するためリストアップする \\
        & \ \  \textbullet \ \ 車が邪魔になりそうな場合は, 駐車場に移動させておく \\
        & \ \  \textbullet \ \ ???は車を駐車場に移動させておく \\                            %後見隊の車の所有者が移動させる
        & \ \  \textbullet \ \ 出発準備が終わったら,出発時間まで????で読み合わせに参加する \\\\     %教室決まったら変更

   8:30 & \textbf{◎ 先遣隊出発} \\
        & \ \  \textbullet \ \ 忘れ物がないかよく確認する \\
        & \ \  \textbullet \ \ ???が報告slackに連絡した後,出発する \\                %先遣隊の報告係が報告
\end{longtable}


\subsection{人員配置(人数により調整,運転者含む)}
\begin{itemize}
\item 先遣隊1:????,????
\item 先遣隊2:????,????
\item 先遣隊3:????,????,????     %多分ここは院生

\end{itemize}

\subsection{備考}
\begin{itemize}
\item 前日までに荷物を????へ運んでおく
\item 荷物は乗せる車ごとに分けておく
\item 研究室へ搬入の際にも物品を確認しておく
\item バス司会(????,????,????,????,????,????)はスタッフ集合部屋(????)で酔い止めと水を受け取る
\end{itemize}


%%%%%%%%%%%%%%%%%%%%%%%%%%%%%%%%%%%%%%%%%%%%%%%%%%%%%%%%%%%%%%%%%%%%%%%%%%%%%%%
%\include{End}
